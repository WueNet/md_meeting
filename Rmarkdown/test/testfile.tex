% Options for packages loaded elsewhere
\PassOptionsToPackage{unicode}{hyperref}
\PassOptionsToPackage{hyphens}{url}
%
\documentclass[
]{article}
\usepackage{amsmath,amssymb}
\usepackage{lmodern}
\usepackage{ifxetex,ifluatex}
\ifnum 0\ifxetex 1\fi\ifluatex 1\fi=0 % if pdftex
  \usepackage[T1]{fontenc}
  \usepackage[utf8]{inputenc}
  \usepackage{textcomp} % provide euro and other symbols
\else % if luatex or xetex
  \usepackage{unicode-math}
  \defaultfontfeatures{Scale=MatchLowercase}
  \defaultfontfeatures[\rmfamily]{Ligatures=TeX,Scale=1}
\fi
% Use upquote if available, for straight quotes in verbatim environments
\IfFileExists{upquote.sty}{\usepackage{upquote}}{}
\IfFileExists{microtype.sty}{% use microtype if available
  \usepackage[]{microtype}
  \UseMicrotypeSet[protrusion]{basicmath} % disable protrusion for tt fonts
}{}
\makeatletter
\@ifundefined{KOMAClassName}{% if non-KOMA class
  \IfFileExists{parskip.sty}{%
    \usepackage{parskip}
  }{% else
    \setlength{\parindent}{0pt}
    \setlength{\parskip}{6pt plus 2pt minus 1pt}}
}{% if KOMA class
  \KOMAoptions{parskip=half}}
\makeatother
\usepackage{xcolor}
\IfFileExists{xurl.sty}{\usepackage{xurl}}{} % add URL line breaks if available
\IfFileExists{bookmark.sty}{\usepackage{bookmark}}{\usepackage{hyperref}}
\hypersetup{
  pdftitle={testfile},
  pdfauthor={Jakob},
  hidelinks,
  pdfcreator={LaTeX via pandoc}}
\urlstyle{same} % disable monospaced font for URLs
\usepackage[margin=1in]{geometry}
\usepackage{color}
\usepackage{fancyvrb}
\newcommand{\VerbBar}{|}
\newcommand{\VERB}{\Verb[commandchars=\\\{\}]}
\DefineVerbatimEnvironment{Highlighting}{Verbatim}{commandchars=\\\{\}}
% Add ',fontsize=\small' for more characters per line
\usepackage{framed}
\definecolor{shadecolor}{RGB}{248,248,248}
\newenvironment{Shaded}{\begin{snugshade}}{\end{snugshade}}
\newcommand{\AlertTok}[1]{\textcolor[rgb]{0.94,0.16,0.16}{#1}}
\newcommand{\AnnotationTok}[1]{\textcolor[rgb]{0.56,0.35,0.01}{\textbf{\textit{#1}}}}
\newcommand{\AttributeTok}[1]{\textcolor[rgb]{0.77,0.63,0.00}{#1}}
\newcommand{\BaseNTok}[1]{\textcolor[rgb]{0.00,0.00,0.81}{#1}}
\newcommand{\BuiltInTok}[1]{#1}
\newcommand{\CharTok}[1]{\textcolor[rgb]{0.31,0.60,0.02}{#1}}
\newcommand{\CommentTok}[1]{\textcolor[rgb]{0.56,0.35,0.01}{\textit{#1}}}
\newcommand{\CommentVarTok}[1]{\textcolor[rgb]{0.56,0.35,0.01}{\textbf{\textit{#1}}}}
\newcommand{\ConstantTok}[1]{\textcolor[rgb]{0.00,0.00,0.00}{#1}}
\newcommand{\ControlFlowTok}[1]{\textcolor[rgb]{0.13,0.29,0.53}{\textbf{#1}}}
\newcommand{\DataTypeTok}[1]{\textcolor[rgb]{0.13,0.29,0.53}{#1}}
\newcommand{\DecValTok}[1]{\textcolor[rgb]{0.00,0.00,0.81}{#1}}
\newcommand{\DocumentationTok}[1]{\textcolor[rgb]{0.56,0.35,0.01}{\textbf{\textit{#1}}}}
\newcommand{\ErrorTok}[1]{\textcolor[rgb]{0.64,0.00,0.00}{\textbf{#1}}}
\newcommand{\ExtensionTok}[1]{#1}
\newcommand{\FloatTok}[1]{\textcolor[rgb]{0.00,0.00,0.81}{#1}}
\newcommand{\FunctionTok}[1]{\textcolor[rgb]{0.00,0.00,0.00}{#1}}
\newcommand{\ImportTok}[1]{#1}
\newcommand{\InformationTok}[1]{\textcolor[rgb]{0.56,0.35,0.01}{\textbf{\textit{#1}}}}
\newcommand{\KeywordTok}[1]{\textcolor[rgb]{0.13,0.29,0.53}{\textbf{#1}}}
\newcommand{\NormalTok}[1]{#1}
\newcommand{\OperatorTok}[1]{\textcolor[rgb]{0.81,0.36,0.00}{\textbf{#1}}}
\newcommand{\OtherTok}[1]{\textcolor[rgb]{0.56,0.35,0.01}{#1}}
\newcommand{\PreprocessorTok}[1]{\textcolor[rgb]{0.56,0.35,0.01}{\textit{#1}}}
\newcommand{\RegionMarkerTok}[1]{#1}
\newcommand{\SpecialCharTok}[1]{\textcolor[rgb]{0.00,0.00,0.00}{#1}}
\newcommand{\SpecialStringTok}[1]{\textcolor[rgb]{0.31,0.60,0.02}{#1}}
\newcommand{\StringTok}[1]{\textcolor[rgb]{0.31,0.60,0.02}{#1}}
\newcommand{\VariableTok}[1]{\textcolor[rgb]{0.00,0.00,0.00}{#1}}
\newcommand{\VerbatimStringTok}[1]{\textcolor[rgb]{0.31,0.60,0.02}{#1}}
\newcommand{\WarningTok}[1]{\textcolor[rgb]{0.56,0.35,0.01}{\textbf{\textit{#1}}}}
\usepackage{graphicx}
\makeatletter
\def\maxwidth{\ifdim\Gin@nat@width>\linewidth\linewidth\else\Gin@nat@width\fi}
\def\maxheight{\ifdim\Gin@nat@height>\textheight\textheight\else\Gin@nat@height\fi}
\makeatother
% Scale images if necessary, so that they will not overflow the page
% margins by default, and it is still possible to overwrite the defaults
% using explicit options in \includegraphics[width, height, ...]{}
\setkeys{Gin}{width=\maxwidth,height=\maxheight,keepaspectratio}
% Set default figure placement to htbp
\makeatletter
\def\fps@figure{htbp}
\makeatother
\setlength{\emergencystretch}{3em} % prevent overfull lines
\providecommand{\tightlist}{%
  \setlength{\itemsep}{0pt}\setlength{\parskip}{0pt}}
\setcounter{secnumdepth}{-\maxdimen} % remove section numbering
\ifluatex
  \usepackage{selnolig}  % disable illegal ligatures
\fi
\usepackage[]{biblatex}
\addbibresource{bibliography.bib}

\title{testfile}
\author{Jakob}
\date{3/25/2021}

\begin{document}
\maketitle

\hypertarget{how-to-start}{%
\subsection{How to Start}\label{how-to-start}}

\begin{enumerate}
\def\labelenumi{\arabic{enumi}.}
\item
  You should have a working version of RStudio (From within RStudio, go
  to Help \textgreater{} Check for Updates to install newer version of
  RStudio (if available).)
\item
  Install and load Rmarkdown with \texttt{install.packages("Rmarkdown")}
  and \texttt{library(rmarkdown)}
\item
  You will be asked to give some information, that information will show
  up in your YAML header (which is some kind of human readable
  non-markup format)

  \begin{itemize}
  \tightlist
  \item
    Important: the header can tell \texttt{render()} what to output
    --\textgreater{} html for example
  \end{itemize}
\end{enumerate}

\hypertarget{basics}{%
\subsection{Basics}\label{basics}}

\hypertarget{flavor}{%
\subsubsection{Flavor}\label{flavor}}

Rmarkdown has not the same ``flavor'' as GitHub Markdown\ldots{} So
unfortunately you will have to use a slightly different syntax for some
things. - you have to use 4 spaces to indent a list element.

\hypertarget{code-blocks}{%
\subsubsection{Code blocks:}\label{code-blocks}}

\begin{itemize}
\tightlist
\item
  create code-blocks as you would do in normal Markdown.
\item
  The difference is, that you can have actual executable code blocks in
  your R-Markdown file and decide if to render everything, just the
  results, just the plots or nothing.
\item
  You have to declare a language in a \texttt{\{r\}} curly brackets
  syntax for this.
\end{itemize}

\begin{verbatim}
` ```{r, include=True, echo=False}
# This is an R code block, which will be included in the rendered output file. 
# ... It would not show the source code but the result/output/plot
# ... you only need three ``` , the first one is necessary here
# ... to escape within the actual code block.
` ```
\end{verbatim}

You can also embed plots, for example:

\includegraphics{testfile_files/figure-latex/pressure-1.pdf}

Note that the \texttt{echo\ =\ FALSE} parameter was added to the code
chunk to prevent printing of the R code that generated the plot.

\begin{verbatim}
## Loading required package: viridisLite
\end{verbatim}

\includegraphics{testfile_files/figure-latex/unnamed-chunk-1-1.pdf}

Or just some table:

\begin{Shaded}
\begin{Highlighting}[]
\FunctionTok{summary}\NormalTok{(cars)}
\end{Highlighting}
\end{Shaded}

\begin{verbatim}
##      speed           dist       
##  Min.   : 4.0   Min.   :  2.00  
##  1st Qu.:12.0   1st Qu.: 26.00  
##  Median :15.0   Median : 36.00  
##  Mean   :15.4   Mean   : 42.98  
##  3rd Qu.:19.0   3rd Qu.: 56.00  
##  Max.   :25.0   Max.   :120.00
\end{verbatim}

\hypertarget{rendering}{%
\subsubsection{Rendering}\label{rendering}}

\begin{itemize}
\item
  Rendering will make a pdf, TeX, HTML, notebook, \ldots{} or
  whatever-you-chose-file from your *.Rmd file.

  \begin{itemize}
  \tightlist
  \item
    You can Render from the command line (or from an R code block) with
    `\texttt{render(markdownfile.Rmd)}
  \item
    To render your current Rmd File, use the shortcut: SHIFT+command+K
    (macOS) or CTRL+SHIFT+K (Windows)
  \end{itemize}
\item
  Now the Header comes in handy, we can tell it to make a Html document
  by writing:

\begin{verbatim}
title: "testfile"
author: "Jakob"
date: "3/25/2021"
output: html_document
\end{verbatim}
\end{itemize}

\hypertarget{get-a-tex-file}{%
\subsubsection{get a TeX file}\label{get-a-tex-file}}

\begin{enumerate}
\def\labelenumi{\arabic{enumi}.}
\setcounter{enumi}{3}
\item
  \begin{itemize}
  \tightlist
  \item
    For example you can use:
    \texttt{\{\}\ title:\ "testfile"\ author:\ "Jakob"\ date:\ "3/25/2021"\ output:\ \ \ pdf\_document:\ \ keep\_tex:\ true\ \ citation\_package:\ biblatex\ bibliography:\ bibliography.bib}
  \end{itemize}
\end{enumerate}

\hypertarget{math-block}{%
\subsubsection{Math Block}\label{math-block}}

\begin{itemize}
\tightlist
\item
  as you know it from Markdown \[
  \longrightarrow c_1 + c_2*h \Sigma^{top}_{bottom}
  \]
\end{itemize}

\printbibliography

\end{document}
